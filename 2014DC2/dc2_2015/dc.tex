\documentclass[11pt,a4paper,twoside]{jarticle}
%==== 科研費LaTeX =============================================
%	2015(H27)年度 DC
%============================================================
% 2008-03-08: Taku Yamanaka (JSPS Research Center for Science Systems / Osaka Univ.)
% 2009-03-03: Yoko Yanagida (Assistant)
% 2010-03-03: Taku: Imported new features introduced in 2009 fall.
% 2011-02-25: Taku: Revised for JFY2013.
% 2013-03-14: Taku: Revised for JFY2014.
%============================================================
%=======================================
% form00_header.tex
%	General header for kakenhiLaTeX,  Moved over from form00_2010_header.tex.
%	2009-09-06 Taku Yamanaka (Osaka Univ.)
%==== General Version History ======================================
% 2006-05-30 Taku Yamanaka (Physics Dept., Osaka Univ.)
% 2006-06-02 V1.0
% 2006-06-14 V1.1 Use automatic calculation for cost tables.
% 2006-06-18 V1.2 Split user's contents and the format.
% 2006-06-20 V1.3 Reorganized user and format files
% 2006-06-25 V1.4 Readjusted all the table column widths with p{...}.
%				With \KLTabR and \KLTabRNum, now the items can be right-justified
%				in the cell defined by p{...}.
% 2006-06-26 V1.5 Use \newlength and \setlength, instead of \newcommand, to define positions.
% 2006-08-19 V1.6 Remade it for 2007 JFY version.
% 2006-09-05 V1.7 Added font declarations suggested by Hoshino@Meisei Univ.
% 2006-09-06 V1.8 Introduced usePDFform flag to switch the form file format.
% 2006-09-09 V1.9 Changed p.7, to allow different heights between years. (Thanks to Ytow.)
% 2006-09-11 V2.0 Added an option to show budget summary.
% 2006-09-13 V2.1 Added an option to show the group.
% 2006-09-14 V2.1.1 Cleaned up Kenkyush Chosho.
% 2006-09-21 V2.2 Generated under a new automatic development system.

% 2007-03-24 V3.0 Switched to a method using "picture" environment.

% 2007-08-14 V3.1 Switched to kakenhi3.sty.
% 2007-09-17 V3.2 Added \KLMaxYearCount
% 2008-03-08 V3.3 Remade it for 2009 JFY version\
% 2008-09-08 V3.4 Added \KLXf ... \KLXh.
% 2011-10-20 V5.0 Use kakenhi5.sty, to utilize array package in tabular environment.
% 2012-08-14 v5.1 Moved preamble and kakenhi5 into the current directory, instead of the parent directory.
% 2012-11-10 v6.0 Switched to kakenhi6.sty.
%=======================================
%============================================================
% preamble.tex
%
% Dummy section and subsection commands.
% With these, some editors (such as TeXShop, etc.) can jump to the (sub)sections.
\newcommand{\dummy}{dummy}% 
\renewcommand{\section}[1]{\renewcommand{\dummy}{#1}}
\renewcommand{\subsection}[1]{\renewcommand{\dummy}{#1}}

% Flag for switching form file format.......
\usepackage{ifthen}
\newboolean{usePDFform}
\newboolean{BudgetSummary}

\usepackage{forms/kakenhi6}

\pagestyle{empty}

% ===== Parameters for LaTeX =========================

% ===== Font declarations  ======================================
\DeclareFontShape{JT1}{mc}{m}{it}{<->ssub * mc/m/n}{}
\DeclareFontShape{JY1}{mc}{m}{it}{<->ssub * mc/m/n}{}

% ===== Parameters for KL (Kakenhi LaTeX) ========================
% general purpose temporary variables	-2007
\newcommand{\KLX}{}
\newcommand{\KLXa}{}
\newcommand{\KLXb}{}
\newcommand{\KLXc}{}
\newcommand{\KLXd}{}
\newcommand{\KLXe}{}
\newcommand{\KLXf}{}
\newcommand{\KLXg}{}
\newcommand{\KLXh}{}
\newcommand{\KLY}{}
\newcommand{\KLYa}{}
\newcommand{\KLYb}{}
\newcommand{\KLXR}{}
\newlength{\KLCella}
\newlength{\KLCellb}
\newlength{\KLCellc}
\newlength{\KLCelld}
\newlength{\KLCelle}
\newlength{\KLCellf}
\newlength{\KLCellg}
\newlength{\KLCellh}

% sub-page
\newlength{\KLSubPageX}
\newlength{\KLSubPageY}
\newlength{\KLspx}
\newlength{\KLspy}
\newcommand{\KLSubPageXmm}{}	% for \input(x,y){....} which uses a unit (mm)
\newcommand{\KLSubPageYmm}{}	% for \input(x,y){....} which uses a unit (mm)

% margins for parbox inside frames; in units of points
\newcounter{KLParboxSideMargin}
\newcounter{KLParboxTopMargin}
\newcounter{KLParboxBottomMargin}

% ===== standard counters ======================================
\newcounter{KLSubPageNo}	% sub-page counter
\newcounter{KLPageOffset}		% to generate sub-page number
\newcounter{KLMaxYearCount}	% # of years for the proposal


% ===== initializations ============
\KLInitTypesettingPageSelection



% user01_header
%=== 様式のファイルの形式の指定 =================
%   PDFではなく、eps の様式を読み込む場合は、次の行の頭に「%」をつけてください。
\setboolean{usePDFform}{true}
%===================================
%==========================================================
% form01_header.tex
%	2014-03-02: Taku Yamanaka (Osaka Univ.)
%		This is called after usePDFform is set.
%		Originally, this part was in form07_header.tex, but then
%		\usepackage{color} that is called before it was not effective.
%		[dvipdfmx] is not used for eps forms, because it makes the forms
%		slightly larger than pdf forms.
%		
%==========================================================
% ===== File format for forms ===========================
\ifthenelse{\boolean{usePDFform}}{
	\newcommand{\KLFormFormat}{pdf}	\usepackage[dvipdfmx]{graphicx}
}{	\newcommand{\KLFormFormat}{eps}	\usepackage{graphicx}
}

%----------------------------------------------------------------------------



% user02_header
%=== 予算の表の印刷 =====================
% 予算の集計の表を出すためには、次の行の頭の%を消してください。
%\setboolean{BudgetSummary}{true}
%=================================

% === 一部のページだけタイプセット ==============
% New in 2009 fall version!
% 選んだページだけタイプセットするには、次の例の頭の%を消し、並べてください。
% 複数のページを選ぶこともできます。
% 提出前には、必ず全てコメントアウト(頭に%をつける)してください。
%ーーーーーーーーーーーーーーーーーーーーーーーーーーーーーーーーー
%\KLTypesetPage{1}			% p.1 (or p.1を含む連続したページ),
%\KLTypesetPage{3}			% p.3 (or p.3を含む連続したページ),
%\KLTypesetPagesInRange{5}{6}	% p.5 ~ p.6,
%\KLTypesetPagesInRange{8}{10}	% and p.8 ~ p.10
%=================================

% ===== my favorite packages ====================================
% ここに、自分の使いたいパッケージを宣言して下さい。
\usepackage{wrapfig}
% \usepackage{amssymb}
%\usepackage{mb}
% \usepackage{color} % でも科研費の書類はグレースケールで印刷されます
%\DeclareGraphicsRule{.tif}{png}{.png}{`convert #1 `dirname #1`/`basename #1 .tif`.png}
%==========================================================

\newcommand{\KLShouKeiLine}[1]{\cline{#1}}
%もし、小計の上の線を取れと事務に言われたら、
%「そのようなことは、記入要項に書かれていないし、学振はそのようなことは気にしていない。」と
% 突っぱねる。
% それでもなお消せと理不尽なことを言われたら、次の行の 最初の「%」を消す。	
%\renewcommand{\KLShouKeiLine}[1]{}

\newcommand{\KLBudgetTableFontSize}{small}	% 予算の表のフォントの大きさ: small, footnotesize
\newcommand{\KLFundsTableFontSize}{normalsize}	%応募中、受入れ予定の研究費のフォントの大きさ:normalsize, small, footnotesize

% ===== my personal definitions ==================================
% ここに、自分のよく使う記号などを定義して下さい。
\newcommand{\klpionn}{K_L \to \pi^0 \nu \overline{\nu}}
\newcommand{\kppipnn}{K^+ \to \pi^+ \nu \overline{\nu}}


% hook3: after including packages ===================
 % for future maintenance
% ===== Global definitions for the PD form ======================
% 基本情報
%
%------ 研究課題名  -------------------------------------------
\newcommand{\研究課題名}{象の卵}

%----- 研究機関名と研究代表者の氏名-----------------------
\newcommand{\研究機関名}{逢坂大学}
\newcommand{\申請者氏名}{湯川秀樹}
\newcommand{\研究代表者氏名}{\申請者氏名}

%---- 研究期間の最終年度 ----------------
\newcommand{\研究期間の最終元号年度}{29}	%平成で、半角数字のみ
%=========================================================
% ===== Global year-dependent definitions for the Kakenhi form ===========
% 基本情報
\newcommand{\研究開始年度}{2015}
\newcommand{\研究開始元号年度}{27}	%平成

\newcommand{\1年目西暦}{2015}
\newcommand{\2年目西暦}{2016}
\newcommand{\3年目西暦}{2017}
\newcommand{\4年目西暦}{2018}
\newcommand{\5年目西暦}{2019}
\newcommand{\6年目西暦}{2020}

\newcommand{\1年目}{27}
\newcommand{\2年目}{28}
\newcommand{\3年目}{29}
\newcommand{\4年目}{30}
\newcommand{\5年目}{31}
\newcommand{\6年目}{32}

\newcommand{\1年目J}{27}
\newcommand{\2年目J}{28}
\newcommand{\3年目J}{29}
\newcommand{\4年目J}{30}
\newcommand{\5年目J}{31}
\newcommand{\6年目J}{32}


	%<<<
%==========================================================
% form03_header.tex
%	2009-03-04: Taku Yamanaka (Osaka Univ.)
%==========================================================
\usepackage{calc}
\usepackage{watermark}
\usepackage{longtable}
\usepackage{geometry}                % See geometry.pdf to learn the layout options. There are lots.
\usepackage{udline}
\usepackage{array}

\geometry{noheadfoot,scale=1}  %scale=1 resets margins to 0
\setlength{\unitlength}{1pt}

% define variables for positions ==========================
% picture environment location, in  units of points
\newcommand{\KLOddPictureX}{}
\newcommand{\KLEvenPictureX}{}
\newcommand{\KLPictureY}{}
\newcommand{\KLOddPictureInWaterMarkX}{}
\newcommand{\KLEvenPictureInWaterMarkX}{}
\newcommand{\KLPictureInWaterMarkY}{}

\newlength{\KLoddsidemargin}
\newlength{\KLevensidemargin}
\newlength{\KLtopmargin}

\newcounter{KLCOddPictureInWaterMarkX}
\newcounter{KLCEvenPictureInWaterMarkX}
\newcounter{KLCPictureInWaterMarkY}
\newcounter{KLCOddPictureX}
\newcounter{KLCEvenPictureX}
\newcounter{KLCPictureY}

%------------------------------------------------------------

\newcommand{\KLLeftEdge}{}
\newcommand{\KLRightEdge}{}

% standard margins for text in frames
\setcounter{KLParboxSideMargin}{7}
\setcounter{KLParboxTopMargin}{12}
\setcounter{KLParboxBottomMargin}{5}

%-----------------------------------------------------------
\newcommand{\KLTwoHLines}{\hline\hline}


%=================================================================
% form05_dc_header.tex
%	for the 2007(H19) Japanese Fiscal Year
%	2006-10-01 : Taku Yamanaka (Osaka Univ.)
%			Switched to the new development system using a "mother file".
%	2007-08-08: Taku
%			Switched to a new method using "picture" environment.
%	2008-03-08: Taku
%			Readjusted parameters for the new 2008 form.
%	2009-09-04: Taku
%			Introduced form03_header and form07_header to automatically calculate margins and
%			other miscellaneous coordinate parameters.
%=================================================================

% ===== Global definitions for the Kakenhi form ======================
% 基本情報
\newcommand{\研究種目}{DC}
\newcommand{\研究種目後半}{}
\ifthenelse{\isundefined{\研究種別}}{
	\newcommand{\研究種別}{}
}{}%
\newcommand{\KLMainFile}{dc.tex}
\newcommand{\KLForms}{dc_forms}
\newcommand{\KLYoshiki}{dc}

% 奇数ページの下に記入される情報
\newcommand{\klbyYup}{}
\newcommand{\klbyYdown}{}
\newcommand{\klbyKikanXleft}{}
\newcommand{\klbyKikanXright}{}
\newcommand{\klbyNameXleft}{}
\newcommand{\klbyNameXright}{}

\newcommand{\KLBottomInfo}[6]{%
	\ifthenelse{\equal{#1}{}}{%
		\renewcommand{\klbyYup}{60}
		\renewcommand{\klbyYdown}{45}
	}{%
		\renewcommand{\klbyYup}{#1}
		\renewcommand{\klbyYdown}{#2}
	}
	
	\ifthenelse{\equal{#3}{}}{%
		\renewcommand{\klbyKikanXleft}{132}
		\renewcommand{\klbyKikanXright}{349}
		\renewcommand{\klbyNameXleft}{425}
		\renewcommand{\klbyNameXright}{550}
	}{%
		\renewcommand{\klbyKikanXleft}{#3}
		\renewcommand{\klbyKikanXright}{#4}
		\renewcommand{\klbyNameXleft}{#5}
		\renewcommand{\klbyNameXright}{#6}
	}
%	\KLTextBox{\klbyKikanXleft}{\klbyYup}{\klbyKikanXright}{\klbyYdown}{}{\研究機関名}%
	\KLTextBox{\klbyNameXleft}{\klbyYup}{\klbyNameXright}{\klbyYdown}{}{\申請者氏名}%
}

%==========================================================
% frame edge positions of multi-page-block
\newcommand{\KLOddMultiPageLeftEdge}{48}
\newcommand{\KLOddMultiPageRightEdge}{550}
\newcommand{\KLEvenMultiPageLeftEdge}{46}
\newcommand{\KLEvenMultiPageRightEdge}{551}

% vertical limits in the first multi-page-block
\newcommand{\KLMultiPageTopEdge}{806}		%lowest top position (except for the 1st page)
\newcommand{\KLMultiPageBottomEdge}{79}	%highest bottom position (except for the last page)

% Modify the edges for single page frames if necessary
\newcommand{\KLOddLeftEdge}{48}
\newcommand{\KLOddRightEdge}{550}
\newcommand{\KLEvenLeftEdge}{46}
\newcommand{\KLEvenRightEdge}{551}

%==========================================================

%

%==========================================================
% form07_header.tex
%	2009-03-04: Taku Yamanaka (Osaka Univ.)
%	2014-03-02: Taku: Moved graphics part to form01_header.tex.
%==========================================================
% Remember Standard Positions that were set in form05_xxxx_header.tex
\let \KLStandardOddMultiPageLeftEdge = \KLOddMultiPageLeftEdge
\let \KLStandardOddMultiPageRightEdge = \KLOddMultiPageRightEdge
\let \KLStandardEvenMultiPageLeftEdge = \KLEvenMultiPageLeftEdge
\let \KLStandardEvenMultiPageRightEdge = \KLEvenMultiPageRightEdge

\let \KLStandardMultiPageTopEdge = \KLMultiPageTopEdge
\let \KLStandardMultiPageBottomEdge = \KLMultiPageBottomEdge

\let \KLStandardOddLeftEdge = \KLOddLeftEdge
\let \KLStandardOddRightEdge = \KLOddRightEdge
\let \KLStandardEvenLeftEdge = \KLEvenLeftEdge
\let \KLStandardEvenRightEdge = \KLEvenRightEdge

%------ This should be set before \begin{document} ------
\KLStandardLengths
\KLStandardPositions
%----------------------------------------------------------------------------


%============================================================
%endPrelude

\begin{document}
% hook5 : right after \begin{document} ==============
 % for future maintenance
%============================================================
%     User Inputs
%============================================================

%form: dc_form_03-04.tex ; user: dc_03-04_preparation_etc.tex
%========== DC =========
%===== p. 03-04 現在までの研究状況 =============
\section{現在までの研究状況}
%watermark: w03_past_dc
\newcommand{\研究の背景}{%
%begin  研究の背景===================
	象の卵の研究の背景は...

	\begin{thebibliography}{99}
		\bibitem{teramura} 寺村輝夫、「ぼくは王様 - ぞうのたまごのたまごやき」.
	\end{thebibliography}
%end  研究の背景 ====================
}

\newcommand{\現在までの研究状況}{%
%begin  現在までの研究状況===================
	今までは、地球上で最大の生物、シロナガスクジラの卵の研究を進めようとしてきた。
	クジラの卵の場合は、高い水圧に耐える必要があるため、堅固の構造となっているはずであり、
	これが解明されれば、将来、深海潜水艇への応用も効く。
	しかし、シロナガスクジラの生息範囲が広い、海に潜っている時間が長い、
	生息数も減っている、などの原因により、
	卵を見つけることができなかった。
	
	そこで、\underline{地球で}最大の動物から、\underline{地上で}最大の動物に研究対象を変更する。

	ぞうの卵はおいしいぞう。
ぞうの卵はおいしいぞう。
ぞうの卵はおいしいぞう。
ぞうの卵はおいしいぞう。
ぞうの卵はおいしいぞう。
ぞうの卵はおいしいぞう。
ぞうの卵はおいしいぞう。
ぞうの卵はおいしいぞう。
ぞうの卵はおいしいぞう。
ぞうの卵はおいしいぞう。
ぞうの卵はおいしいぞう。
ぞうの卵はおいしいぞう。
ぞうの卵はおいしいぞう。
ぞうの卵はおいしいぞう。
ぞうの卵はおいしいぞう。
ぞうの卵はおいしいぞう。
ぞうの卵はおいしいぞう。
ぞうの卵はおいしいぞう。
ぞうの卵はおいしいぞう。
ぞうの卵はおいしいぞう。
ぞうの卵はおいしいぞう。
ぞうの卵はおいしいぞう。
ぞうの卵はおいしいぞう。
ぞうの卵はおいしいぞう。
ぞうの卵はおいしいぞう。
ぞうの卵はおいしいぞう。
ぞうの卵はおいしいぞう。
ぞうの卵はおいしいぞう。
ぞうの卵はおいしいぞう。
ぞうの卵はおいしいぞう。
ぞうの卵はおいしいぞう。
ぞうの卵はおいしいぞう。
ぞうの卵はおいしいぞう。
ぞうの卵はおいしいぞう。
ぞうの卵はおいしいぞう。
ぞうの卵はおいしいぞう。
ぞうの卵はおいしいぞう。
ぞうの卵はおいしいぞう。
ぞうの卵はおいしいぞう。
ぞうの卵はおいしいぞう。
ぞうの卵はおいしいぞう。
ぞうの卵はおいしいぞう。
ぞうの卵はおいしいぞう。
ぞうの卵はおいしいぞう。
ぞうの卵はおいしいぞう。
ぞうの卵はおいしいぞう。
ぞうの卵はおいしいぞう。
ぞうの卵はおいしいぞう。
ぞうの卵はおいしいぞう。
ぞうの卵はおいしいぞう。
ぞうの卵はおいしいぞう。
ぞうの卵はおいしいぞう。
ぞうの卵はおいしいぞう。
ぞうの卵はおいしいぞう。
ぞうの卵はおいしいぞう。
ぞうの卵はおいしいぞう。
ぞうの卵はおいしいぞう。
ぞうの卵はおいしいぞう。
ぞうの卵はおいしいぞう。
ぞうの卵はおいしいぞう。
ぞうの卵はおいしいぞう。
ぞうの卵はおいしいぞう。
ぞうの卵はおいしいぞう。
ぞうの卵はおいしいぞう。
ぞうの卵はおいしいぞう。
ぞうの卵はおいしいぞう。
ぞうの卵はおいしいぞう。
ぞうの卵はおいしいぞう。
ぞうの卵はおいしいぞう。
ぞうの卵はおいしいぞう。
ぞうの卵はおいしいぞう。
ぞうの卵はおいしいぞう。
ぞうの卵はおいしいぞう。
ぞうの卵はおいしいぞう。
ぞうの卵はおいしいぞう。
ぞうの卵はおいしいぞう。
ぞうの卵はおいしいぞう。
ぞうの卵はおいしいぞう。
ぞうの卵はおいしいぞう。
ぞうの卵はおいしいぞう。
ぞうの卵はおいしいぞう。
ぞうの卵はおいしいぞう。
ぞうの卵はおいしいぞう。
ぞうの卵はおいしいぞう。
ぞうの卵はおいしいぞう。
ぞうの卵はおいしいぞう。
ぞうの卵はおいしいぞう。
ぞうの卵はおいしいぞう。
ぞうの卵はおいしいぞう。
ぞうの卵はおいしいぞう。
ぞうの卵はおいしいぞう。
ぞうの卵はおいしいぞう。
ぞうの卵はおいしいぞう。
ぞうの卵はおいしいぞう。
ぞうの卵はおいしいぞう。
ぞうの卵はおいしいぞう。
ぞうの卵はおいしいぞう。
ぞうの卵はおいしいぞう。
ぞうの卵はおいしいぞう。
ぞうの卵はおいしいぞう。
ぞうの卵はおいしいぞう。
ぞうの卵はおいしいぞう。
ぞうの卵はおいしいぞう。
ぞうの卵はおいしいぞう。
ぞうの卵はおいしいぞう。
ぞうの卵はおいしいぞう。
ぞうの卵はおいしいぞう。
ぞうの卵はおいしいぞう。
ぞうの卵はおいしいぞう。
ぞうの卵はおいしいぞう。
ぞうの卵はおいしいぞう。
ぞうの卵はおいしいぞう。
ぞうの卵はおいしいぞう。
ぞうの卵はおいしいぞう。
ぞうの卵はおいしいぞう。
ぞうの卵はおいしいぞう。
ぞうの卵はおいしいぞう。
ぞうの卵はおいしいぞう。
ぞうの卵はおいしいぞう。
ぞうの卵はおいしいぞう。
ぞうの卵はおいしいぞう。
ぞうの卵はおいしいぞう。
ぞうの卵はおいしいぞう。
ぞうの卵はおいしいぞう。
ぞうの卵はおいしいぞう。
ぞうの卵はおいしいぞう。
ぞうの卵はおいしいぞう。
ぞうの卵はおいしいぞう。
ぞうの卵はおいしいぞう。
ぞうの卵はおいしいぞう。
ぞうの卵はおいしいぞう。
ぞうの卵はおいしいぞう。
ぞうの卵はおいしいぞう。
ぞうの卵はおいしいぞう。
ぞうの卵はおいしいぞう。
ぞうの卵はおいしいぞう。
ぞうの卵はおいしいぞう。
ぞうの卵はおいしいぞう。
ぞうの卵はおいしいぞう。
ぞうの卵はおいしいぞう。
ぞうの卵はおいしいぞう。
  % << only for demonstration. Please delete it or comment it out.	
%end  現在までの研究状況 ====================
}

%form: dc_form_05.tex ; user: dc_05_purpose.tex
%========== DC =========
%===== p. 05 研究の目的・内容 =============
\subsection{研究の目的・内容}
\newcommand{\研究目的}{%
%begin  研究目的と内容===================
	象の卵の研究の目的は...
%end  研究目的と内容 ====================
}

%form: dc_form_06.tex ; user: dc_06_plan.tex
%========== DC =========
%===== p. 06 研究の特色・独創的な点、年次計画 =============
\newcommand{\研究の特色と独創的な点}{%
%begin  研究の特色と独創的な点===================
	象の卵の特色と独創的な点は...
%end  研究の特色と独創的な点 ====================
}

\newcommand{\年次計画1年目}{%
%begin  年次計画1年目===================
	初年度は、まず世界の動物園を巡り、象舍に卵が隠されていないか、探す。
	\vspace{3cm}% adjust the length if necessary
%end  年次計画1年目 ====================
}

\newcommand{\年次計画2年目}{%
%begin  年次計画2年目===================
	2年目はアフリカに行き、空と地上から象の卵を探す。
	アフリカ象は気性が荒いが、サバンナの方がジャングルよりも見通しが効くので、
	インドよりもアフリカを先に探索する。
	\vspace{3cm}% adjust the length if necessary
%end  年次計画2年目 ====================
}

\newcommand{\年次計画3年目}{%
%begin  年次計画3年目===================
	3年目は、インドとタイに行き、ジャングルに隠されている卵を探す。
	ジャングルの場合は空からは探しにくいが、象使いも多く、象の背中に乗って
	象の視点から探索することができる。
	さらに、気だての優しいインド象ならば
	卵の在処を教えてくれる可能性もある。
	\vspace{3cm}% adjust the length if necessary
%end  年次計画3年目 ====================
}

%form: dc_form_07.tex ; user: dc_07_rights.tex
%========== DC =========
%===== p. 07 人権の保護及び法令等の遵守への対応 =============
\subsection{人権の保護及び法令等の遵守への対応}
\newcommand{\人権の保護及び法令等の遵守への対応}{%
%begin  人権の保護及び法令等の遵守への対応 ===================
	象の卵のES細胞の培養、象のクローンの生成などは行わない。
	象個体を現地から持ち出すことはないので、ワシントン条約ならびに
        生物多様性条約に抵触しない。また、組換え実験は行なわないので、
        カルタヘナ議定書にも抵触しない。
%end  人権の保護及び法令等の遵守への対応 ====================
}

%form: dc_form_08.tex ; user: dc_08_publications.tex
%========== DC =========
%===== p. 08 研究業績 =============
\section{研究業績}
\subsection{学術雑誌(紀要・論文集等も含む)に発表した論文及び著書}
\newcommand{\学術雑誌等に発表した論文または著書}{%
%begin  学術雑誌等に発表した論文または著書===================
	
	\begin{enumerate}
		\item[](査読有り)%===========================
		\item \underline{H. Yukawa}$^1$, J. Kara$^2$,
				``Theory of Elephant Eggs'', 
				Phys.\ Rev.\ Lett. {\bf 800}, 800-804 (2005). 
				
		\item F.~Ehrlich, \underline{H. Yukawa}$^1$,
				``You can't Lay an Egg If You're an Elephant'', 
				JofUR\\
				 ({\tt www.universalrejection.org}), {\bf N/A}, N/A (2002).

		\item[](査読なし)%=============================
		\item Kobo Abe$^3$, \underline{H. Yukawa}$^1$, 
				``仔象は死んだ'', 
				安部公房全集, {\bf 26}, 100-200, (2004).
	\end{enumerate}
	他5報\\
	注:著者の所属・職(論文発表時)\\
	1: 大阪大学理学研究科大学院生、
	2: 明治大学文学部学生、
	3: 東京帝国大学医学部学生、
	4: ミッキー大学教授、
	5: Univ. of Zoo大学院生
%end  学術雑誌等に発表した論文または著書 ====================
}

\subsection{学術雑誌等又は商業誌における解説・総説}
\newcommand{\学術雑誌等または商業誌における解説や総説}{%
%begin  学術雑誌等または商業誌における解説や総説===================
	\begin{enumerate}
		\item R.~Kipling, \underline{H. Yukawa},
				``The Elephant's Child (象の鼻はなぜ長い)'', 
				Nature, {\bf 999}, 777-779, (2003).
	\end{enumerate}
	他2件
%end  学術雑誌等または商業誌における解説や総説 ====================
}

\subsection{国際会議における発表}
\newcommand{\国際会議における発表}{%
%begin  国際会議における発表===================
	\begin{enumerate}
		\item $\circ$ 湯川秀樹、
			``Theory of Elephant Eggs'', 
			原始殻物理国際会議、
			カラチ、2006年2月

		\item $\circ$ 湯川秀樹、Jacques-Yves Cousteau,
			``How to search for whale eggs'',
			国際海洋探索会議、ハワイ、2003年4月
	\end{enumerate}
	他1件
%end  国際会議における発表 ====================
}

\subsection{国内学会・シンポジウムにおける発表}
\newcommand{\国内学会やシンポジウムにおける発表}{%
%begin  国内学会やシンポジウムにおける発表===================
	\begin{enumerate}
		\item $\circ$ 湯川秀樹、朝永振一郎、
			「ほ乳類の真の意味」、
			ほ乳類学会、
			東京、2003年6月
	\end{enumerate}
	他3件
%end  国内学会やシンポジウムにおける発表 ====================
}

\subsection{特許等}
\newcommand{\特許等}{%
%begin  特許等===================
	\begin{enumerate}
		\item[](公開中)
		\item 800800号、「クジラの卵を用いた深海潜水艇」\underline{湯川秀樹}、2003年4月
%		\item[] (申請中)
%		\item 8000000号、「象の卵を用いた(ひ・み・つ)」、\underline{湯川秀樹}、2007年4月
	\end{enumerate}		
%end  特許等 ====================
}

\subsection{その他の業績}
\newcommand{\その他の業績}{%
%begin  その他の業績===================
		\begin{enumerate}
			\item もうすぐもらえるで賞
		\end{enumerate}
%end  その他の業績 ====================
}

\subsection{発表前}
\newcommand{\発表前の業績}{%
%begin  発表前の業績===================
	【発表(印刷)前】\\
	{\bf(1)学術雑誌等(紀要・論文集等も含む)に採録決定されたもの}\\
	 (査読有り)\\
	 1) 島崎藤村$^2$ 「夜明け前」『中央公論』中央公論社、800号、pp1-800 (1935). 
	  (証明書\textcircled{1}添付)\\
	  \noindent
	  {\bf(4)国際学会・シンポジウム等における発表の申し込みが受理されたもの}\\
	 2) ◯\underline{湯川秀樹}「象の卵、ついに発見!」『オリンピック』ロンドン, 2012年
	  (証明書\textcircled{2}添付)
%end  発表前の業績 ====================
}

%form: dc_form_09.tex ; user: dc_09_myself.tex
%========== DC =========
%===== p. 09 自己評価 =============
\section{自己評価}
\newcommand{\自己評価}{%
%begin  自己評価===================
{\bf 1. 研究職を志望する動機、目指す研究者像、自己の長所等}

私は、今までの世界観を全く変える発見をしたい。
そのためには、あくせく金を稼ぐ普通の仕事ではなく、
じっくりと研究と野球に取り組める研究職しかない。

私が理想とする研究者は、三四郎の友人でもある、野々宮宗八である。
彼は俗世間の現実に煩わされることなく、
地下の実験室で黙々と光の圧力の測定に取り組んでいる。
彼こそが、真の研究者である。

自分の長所は、まじめで賢いことである。

\vspace{5mm}
{\bf 2. 自己評価をする上で、特に重要と思われる事項}

間もなく、ノーベル賞受賞予定。
%end  自己評価 ====================
}


%endUserFiles
% hook7 : right before including forms ============
 % for future maintenance

% dc_forms
%=======================================
\ifthenelse{\boolean{BudgetSummary}\OR\boolean{klTypesetPage0}}{
	%============================================================
%  Warning cover page
%============================================================

\begin{picture}(0,0)(\KLOddPictureX,\KLPictureY)
	\KLParbox{100}{700}{550}{600}{t}{
		\LARGE
		提出前に次の行を以下のようにコメントアウトし、\\
		コンパイルし直してください。\\
		\hspace{2cm}\%\textbackslash setboolean\{BudgetSummary\}\{true\}\\
		\hspace{2cm}\%\textbackslash KLTypesetPage\{..\}\\
		\hspace{2cm}\%\textbackslash KLTypesetPagesInRange\{..\}\{..\}\\
	}
	\西暦
	\KLParbox{100}{550}{500}{500}{t}{
		\begin{center}
			\LARGE 予算と研究組織のまとめ \\
			\Large \today
		\end{center}
	}

	\KLTextBox{100}{500}{550}{300}{}{
		\Large
		研究種目: \研究種目\研究種別\研究種目後半\\
		研究期間: \研究開始年度(H\研究開始元号年度) 〜 H\研究期間の最終元号年度\\
		研究課題名:「\研究課題名」\\
		研究代表者:\研究代表者氏名\\
		研究機関名:\研究機関名\\
	}
\end{picture}
\clearpage


}{}

\KLInputIfPageInRangeIsSelected{1}{2}{forms/dc_form_03-04}
\KLInputIfSelected{3}{forms/dc_form_05}
\KLInputIfSelected{4}{forms/dc_form_06}
\KLInputIfSelected{5}{forms/dc_form_07}
\KLInputIfSelected{6}{forms/dc_form_08}
\KLInputIfSelected{7}{forms/dc_form_09}
\KLInputIfSelected{8}{forms/dc_form_10}

%========================================


%endFormatFile

% hook9 : right before \end{document} ============
 % for future maintenance
\end{document}
